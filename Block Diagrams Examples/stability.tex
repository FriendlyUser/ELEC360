\documentclass[12pt]{article}

%\usepackage{html}
\begin{document}
\begin{center}
{\LARGE \bf Stability Analysis}
\end{center}

\section*{Locations of poles}

As discussed previously, given the transfer function of a system:
\[
T(s)=\frac{N(s)}{D(s)}=\frac{\sum_{k=0}^m b_k s^k}{\sum_{k=0}^n a_k s^k}
=\frac{\prod_{k=1}^m (s-s_{z_k})}{\prod_{k=1}^n (s-s_{p_k})} 
\]
(where $a_0$ is always assumed to be 1 without loss of generality), we 
can determine whether it is stable by checking if all of its poles $p_i's$
$(i=1,2,\cdots,n)$ (roots of the characteristic polynomial $D(s)$) are 
located on the left plane (LP) of the S-plane, i.e., the real part of each
pole is negative:
\[ Re[p_i]<0 \]
This criterion of stability is easy to understand and test, but it requires
finding the roots of the denominator polynomial, which may not be easy when
the order $n$ is higher than two. It is therefore desirable to determine
the stability of the system based on the coefficients of the denominator
without solving it to find its roots. 

One necessary condition for stability is all coefficients of $D(s)$ are 
positive $a_k>0$ (most obvious for $n<3$). Although this condition is easy
to check, it is not sufficient, i.e., the system may still be unstable even 
if $a_k>0$ for all $k=1,2,\cdots,n$. We need to find a necessary and 
sufficient condition.

\section*{Routh's criterion}

Routh (1874) developed a a necessary and sufficient condition for stability
based on Routh array, which states:

{\bf Routh's criterion: } 
A system is stable if and only if all the elements in the first column of the
Routh array are possitive.

{\bf Routh array: } The first two rows of the Routh array are composed of
the even and odd coefficients of the characteristic polynomial, respectively,
while the remaining rows are composed of elements derived from the first
two rows:
\[ \begin{array}{lllll}
\mbox{row n}   & a_0 & a_2 & a_4 & \cdots \\
\mbox{row n-1} & a_1 & a_3 & a_5 & \cdots \\
\mbox{row n-2} & b_1 & b_2 & b_3 & \cdots \\
\mbox{row n-3} & c_1 & c_2 & c_3 & \cdots \\
\cdots & \cdots & \cdots & \cdots & \cdots \\
\mbox{row 2} & * & * &  & \cdots \\
\mbox{row 1} & * &   &  & \cdots \\
\mbox{row 0} & * &   &  & \cdots \end{array} \]
The elements of the third row are computed as follows:
\[ b_1=-\frac{det\left[\begin{array}{cc}a_0&a_2\\a_1&a_3\end{array}\right]}{a_1}
=\frac{a_1a_2-a_0a_3}{a_1} \]
\[ b_2=-\frac{det\left[\begin{array}{cc}a_0&a_4\\a_1&a_5\end{array}\right]}{a_1}
=\frac{a_1a_4-a_0a_5}{a_1} \]
\[ b_3=-\frac{det\left[\begin{array}{cc}a_0&a_6\\a_1&a_7\end{array}\right]}{a_1}
=\frac{a_1a_6-a_0a_7}{a_1} \]
The elements of the forth row are computed as follows:
\[ c_1=-\frac{det\left[\begin{array}{cc}a_1&a_3\\b_1&b_2\end{array}\right]}{b_1}
=\frac{b_1a_3-a_1b_2}{b_1} \]
\[ c_2=-\frac{det\left[\begin{array}{cc}a_1&a_5\\b_1&b_3\end{array}\right]}{b_1}
=\frac{b_1a_5-a_1b_3}{b_1} \]
\[ c_3=-\frac{det\left[\begin{array}{cc}a_1&a_7\\b_1&b_4\end{array}\right]}{b_1}
=\frac{b_1a_7-a_1b_4}{b_1} \]
The elements from the third row on are computed based on the determinant of a
2 by 2 array composed of the two elements of the first column of the previous
two rows and the two elements of the subsequent columns. Any missing coefficient
is represented by a zero.

{\bf Example 0: }
\[ D(s)=a_0s^3+a_1s^2+a_2s+a_3=0 \]
\[ \begin{array}{rrr}
\mbox{row 3} & a_0 & a_2 \\
\mbox{row 2} & a_1 & a_3 \\
\mbox{row 1} & (a_2a_1-a_0a_3)/a_1 & \\
\mbox{row 0} & a_3 & 
\end{array} \]
For this system to be stable, we must have $a_2a_1-a_0a_3>0$. 

{\bf Example 1: }
\[ D(s)=s^4+2s^3+3s^2+4s+5=0 \]
\[ \begin{array}{rrrr}
\mbox{row 4}  & 1 & 3 & 5 \\
\mbox{row 3}  & 2 & 4 & 0 \\
\mbox{row 3'} & 1 & 2 & 0 \\
\mbox{row 2}  & 1 & 5 &   \\
\mbox{row 1}  & -3 &  &   \\
\mbox{row 0}  & 5  &  &   
\end{array} \]
There are two sign changes indicating two poles on RP. Note that row 3 is divided
by 2 to become row 3' without affecting the result.

{ Example 2: }

\[ D(s)=s^6+4s^5+3s^4+2s^3+s^2+4s+4 \]
with $a_0=1, a_1=4, a_2=3, a_3=2, a_4=1, a_5=4, a_6=4$. First we find the Routh 
array:
\[ \begin{array}{rrrrr}
\mbox{row 6} & 1 & 3 & 1 & 4 \\
\mbox{row 5} & 4 & 2 & 4 & 0 \\
\mbox{row 4} & (4\cdot 3-1\cdot 2)/4=2.5 & (4\cdot 1-4\cdot 1)/4=0 & (4\cdot 4-1\cdot 0)/4=4 & \\
\mbox{row 3} & (2.5\cdot 2-4\cdot 0)/2.5=2 & (2.5\cdot 4-4\cdot 4)/2.5=-2.4 & 0 & \\
\mbox{row 2} & (2\cdot 0-2.5\cdot (-2.4))/2=3 & (2\cdot 4-2.5\cdot 0)=4 & 0 & \\
\mbox{row 1} & (3\cdot (-2.4)-2\cdot 4)/3=-76/15 &  &  & \\
\mbox{row 0} & (-76/15\cdot 4-0\cdot 3)/(-76/15)=4 &  &  & 
\end{array} \]
The elements in the first column are: 1, 4, 2.5, 2, 3, -76/15, 4 with two sign 
changes (3 to -76/15 and -76/15 to 4), there are two poles on the RP and the
system is not stable. Solving the characteristic equation, we can get the five
roots: $-3.26, 0.68\pm 0.75i, -0.60\pm 0.99i,-0.89$.

{\bf Example 3:}

The transfer function of the feedforward pass of a feedback system is 
\[ H(s)=K\;\frac{s+1}{s(s-1)(s+6)} \]
and the feedback gain is $G(s)=-1$ (negative feedback). The overall transfer 
function is therefore:
\[ T(s)=\frac{H(s)}{1-H(s)G(s)}=\frac{K\;\frac{s+1}{s(s+1)(s+2)}}{1+\frac{s+1}{s(s+1)(s+2)}}
=\frac{K(s+1)}{s(s-1)(s+6)+K(s+1)} \]
The characteristic equation is
\[ s^3+5 s^2+(K-6) s+K=0 \]
Routh's criterion is used To find the range of the gain $K$ for stability:
\[ \begin{array}{rrr}
\mbox{row 3} & 1 & K-6 \\
\mbox{row 2} & 5 & K   \\
\mbox{row 1} & (4K-30)/5 & 0 \\
\mbox{row 0} & K & 0 
\end{array} \]
For all elements of the first column to be positive, we have $K>0$ and $K>7.5$.
Consider the following three cases:
\begin{itemize}
\item When $K=9>7.5$, the roots of the characteristic equation are
$p_0=-4.77$, $p_{1,2}=-0.17\pm 1.37$, representing respectively an exponentially 
decaying term and a sinusoid also exponentially decaying.

\item When $K=7.5$, the characteristic equation is
\[ D(s)=s^3+5s^2+1.5s+7.5=(s+5)(s^2+1.5)=(s+5)(s+j\sqrt{1.5})(s-j\sqrt{1.5})=0 \]
with three poles $p_0=-5$ representing an exponential decay and a complex conjugate
pair $p_{1,2}=\pm i\sqrt{1.5}$ representing a sinusoidal oscillation.

\item When $K=5<7.5$, the roots of the characteristic equation are
$p_0=-5.36$, $p_{1,2}=0.18\pm 0.95 i$, representing respectively an exponentially 
decaying term and a sinusoid which grows exponentially.
\end{itemize}

{\bf Example 4:}
Consider a PI controller $K+K_I/s$ with gains $K$ and $K_I$ controlling a plant
$1/(s+1)(s+2)$ as shown below. The closed-loop transfer function is
\[ T(s)=\frac{H(s)}{1+G(s)H(s)}=\frac{(K+\frac{K_I}{s})\frac{1}{(s+1)(s+2)}}{1+(K+\frac{K_I}{s})\frac{1}{(s+1)(s+2)}}
=\frac{sK+K_I}{s(s+1)(s+2)+sK+K_I}
\]
The characteristic equaion is
\[ s(s+1)(s+2)+sK+K_I=s^3+3s^2+(K+2)s+K_I=0 \]
The Routh array is
\[ \begin{array}{rrr}
\mbox{row 3} & 1 & K+2 \\
\mbox{row 2} & 3 & K_I \\
\mbox{row 1} & (6+3K-K_I)/3 &  \\
\mbox{row 0} & K_I & 
\end{array} \]
For the system to be stable, we must have $K_I>0$ and $K>K_I/3-2$.





{\bf Example 5 (homework):}

The transfer function of the feedforward pass of a feedback system is 
\[ H(s)=\frac{K}{s(s+1)(s+2)} \]
and the feedback gain is just $G(s)=-1$ (negative feedback). The overall transfer 
function is therefore:
\[ T(s)=\frac{H(s)}{1-H(s)G(s)}=\frac{K}{s(s+1)(s+2)+K} \]
and the characteristic equation is
\[ D(s)=s^3+3s^2+2s+K=0 \]
Routh's criterion is used To find the range of the gain $K$ for stability:
\[ \begin{array}{rrr}
\mbox{row 3} & 1 & 2 \\
\mbox{row 2} & 3 & K \\
\mbox{row 1} & (6-K)/3 & 0 \\
\mbox{row 0} & K &  \\
\end{array} \]
For all elements of the first column to be positive, we have $K>0$ and $K<6$, i.e.,
$0<K<6$. Consider the following five cases:
\begin{itemize}
\item When $K=-1<0$, we have
\[ T(s)=\frac{-1}{s^3+3s^2+2s-1}=\frac{-1}{(s-0.32)(s+1.66+0.56i)(s+1.66-0.56i)}\]
with three poles $p_0=0.32$ representing an exponentially growing term, and
$p_{1,2}=1.66\pm 0.56i$ representing a decaying sinusoid. The system is unstable.
\item When $K=0$, we have
\[ T(s)=\frac{0}{s^3+3s^2+2s}=0 \]
\item When $0<K=3<6$, we have
\[ T(s)=\frac{3}{s^3+3s^2+2s+3}=\frac{6}{(s+3)(s+i\sqrt{2})(s-i\sqrt{2})} \]
with three poles $p_0=-2.67$ representing an exponential decay and a complex conjugate
pair $p_{1,2}=-0.16\pm 1.05i$ representing a decaying sinusoid. The system is stable.
\item When $K=6$, we have
\[ T(s)=\frac{6}{s^3+3s^2+2s+6}=\frac{6}{(s+3)(s+i\sqrt{2})(s-i\sqrt{2})} \]
with three poles $p_0=-3$ representing an exponential decay and a complex conjugate
pair $p_{1,2}=\pm i\sqrt{2}$ representing a sinusoidal oscillation.
\item When $K=8>6$, we have
\[ T(s)=\frac{8}{s^3+3s^2+2s+8}=\frac{8}{(s-3.17)(s-0.08+1.59i)(s-0.08-1.59i)} \]
with three poles $p_0=3.17$ representing an exponentially growing term and a complex 
conjugate pair $p_{1,2}=-0.08 \pm i\sqrt{2}$ representing a decaying sinusoid. The 
system is unstable.
\end{itemize}


{\bf Special Case 1:}

If the first column contains a zero, the elements in the following row cannot be
evaluated (divided by zero). In this case, the zero will be replaced by a small
value $\epsilon>0$ and the elements of the following rows can be obtained in 
terms of $\epsilon$. At the end, we take the limit $\epsilon\rightarrow 0$.

{\bf Example 6:}
\[ D(s)=s^5+3s^4+2s^3+6s^2+6s+9=0 \]
\[ \begin{array}{lrrr}
\mbox{row 5} & 1 & 2 & 6 \\
\mbox{row 4} & 3 & 6 & 9 \\
\mbox{row 4'}& 1 & 2 & 3 \\
\mbox{row 3} & 0 & 3 &   \\
\mbox{row 3'}& \epsilon & 3 &   \\
\mbox{row 2} & (2\epsilon-3)/\epsilon & 3 &  \\
\mbox{row 1} & 3-3\epsilon^2/(2\epsilon-3) & 0 &  \\
\mbox{row 0} & 3 & 0 &  \\
\end{array} \]
Taking the limit $\epsilon\rightarrow 0$, the first element in row 2 is negative,
while the first element in row 1 is positive, i.e., there are two sign changes
in the first column indicating there are two roots on the RP. In fact, the five
roots are $-2.9, 0.66\pm 1.29i, -0.70\pm 0.99$.

{\bf Special Case 2:}

It is possible that all elements of a row, assumed to be the ith row, are zero. 
In such case, the rest of the Routh array can still be obtained by the following
steps:
\begin{enumerate}
\item treat the previous (i+1)th row as an auxiliary polynomial $P(s)$
\item take derivative of $P(s)$, and use the resulting coefficients as the
  elements of the ith row
\item continue as normal to obtain the rest elements of the array.
\end{enumerate}

{\bf Example 1: }
\[ D(s)=s^5+2s^4+24s^3+48s^2-25s-50=0 \]
\[ \begin{array}{lrrr}
\mbox{row 5} & 1 & 24 & -25 \\
\mbox{row 4} & 2 & 48 & -50 \\
\mbox{row 4'}& 1 & 24 & -25 \\
\mbox{row 3} & 0 &  0 &     \\
\mbox{row 3'}& 4 &  48 &    \\
\mbox{row 3''}& 1 &  12 &   \\
\mbox{row 2} & 24 & -50 &   \\
\mbox{row 1} & 14.08 & 0 &  \\
\mbox{row 0} & -50  &  &    
\end{array} \]
As all elements in row 3 are zero, an auxiliary polynomial is formed using the elements
of the previoius row and its derivative is obtained:
\[ P(s)=s^4+24s^2-25,\;\;\;\;\;\frac{d}{ds}P(s)=4s^3+48s \]
The coefficients of $P'(s)$ are then used as the elements of the ith row and the process
continues as normal. As there is only one sign change in the first column, there is one
root on the RP.

The auxiliary polynomial $P(s)=s^4+24s^2-25=(s^2-1)(s^2+25)$ can be solved to find its
roots $s=\pm 1$ and $s=\pm i5$. They always lie radially opposite in the S-plane. the 
original characteristic polynomial can be factored as
\[ D(s)=s^5+2s^4+24s^3+48s^2-25s-50=(s^4+24s^2-25)(s+2)=0 \]
It can be seen that the roots of $D(s)$ include those of $D(s)$, with an additional
one $s=-2$.

{\bf Example 2: }
\[ D(s)=s^5+5s^4+11s^3+23s^2+28s+12=0 \]
\[ \begin{array}{lrrr}
\mbox{row 5} & 1 & 11 & 28 \\
\mbox{row 4} & 5 & 23 & 12 \\
\mbox{row 3} & 6.4 & 25.6 &\\
\mbox{row 2} & 3 & 12 &    \\
\mbox{row 1} & 0 & 0  &    \\
\mbox{row 1'}& 6 & 0  &    \\
\mbox{row 0} & 12 &   &    
\end{array} \]
where the auxiliary polynomial is $3s^2+12$ and its derivative is $6s$. There is
no sign changes in the first column, therefore all roots of $D(s)$ are on the LP.
The roots of the auxiliary polynomial is $3s^2+12=3(s^2+4)$ are $s=\pm 2i$ (on the
opposit sides of the origin). The original polynomial can be factored as
\[ s^5+5s^4+11s^3+23s^2+28s+12=(s^2+4)(s+1)^2(s+3)=0 \]
with roots $-1$, $-1$, $-3$ in addition to the roots $\pm 2i$ of $P(s)$.


\end{document}
