
% This LaTeX was auto-generated from MATLAB code.
% To make changes, update the MATLAB code and republish this document.

\documentclass{article}
\usepackage{graphicx}
\usepackage[
HomeHTMLFilename=index,     % Filename of the homepage.
HTMLFilename={node-},       % Filename prefix of other pages.
IndexLanguage=english,      % Language for xindy index, glossary.
latexmk,                    % Use latexmk to compile.
%   OSWindows,                  % Force Windows. (Usually automatic.)
mathjax,                    % Use MathJax to display math.
]{lwarp}
\usepackage{color}

\sloppy
\definecolor{lightgray}{gray}{0.5}
\setlength{\parindent}{0pt}


\begin{document}

    
    
\section*{Midterm PREP Script}

\begin{par}
Chapter 6 --- Stabability Question 1, Nise Textbook
\end{par} \vspace{1em}

\subsection*{Contents}

\begin{itemize}
\setlength{\itemsep}{-1ex}
   \item Question 12
   \item Quesition 20
   \item Question 21
   \item Question 22
   \item Example 7.2 from Nise
   \item Example 7.3 from Nise
   \item Example 7.4 from Nise
   \item Chapter 7 or 8 root Locus Plots
\end{itemize}
\begin{par}
s\^{}5 +3s\^{}4+5s\^{}3 + 4s\^{}2+s+3
\end{par} \vspace{1em}
\begin{verbatim}
polyVector = [1 3 5 4 1 3]
RouthHurwitz(polyVector)
\end{verbatim}

        \color{lightgray} \begin{verbatim}
polyVector =

     1     3     5     4     1     3


 Routh-Hurwitz Table:

rhTable =

    1.0000    5.0000    1.0000
    3.0000    4.0000    3.0000
    3.6667         0         0
    4.0000    3.0000         0
   -2.7500         0         0
    3.0000         0         0

~~~~~> it is an unstable system! <~~~~~

 Number of right hand side poles = 2

 Given polynomial coefficients roots :

sysRoots =

  -1.6313 + 0.0000i
  -0.9407 + 1.5042i
  -0.9407 - 1.5042i
   0.2563 + 0.7201i
   0.2563 - 0.7201i


ans =

    1.0000    5.0000    1.0000
    3.0000    4.0000    3.0000
    3.6667         0         0
    4.0000    3.0000         0
   -2.7500         0         0
    3.0000         0         0

\end{verbatim} \color{black}
    \begin{par}
Q2
\end{par} \vspace{1em}
\begin{verbatim}
v2 = [1 0 6 5 8 20]
RouthHurwitz(v2)
\end{verbatim}

        \color{lightgray} \begin{verbatim}
v2 =

     1     0     6     5     8    20


 Routh-Hurwitz Table:

rhTable =

     1     6     8
     0     5    20
  -Inf  -Inf     0
   NaN   NaN     0
   NaN   NaN     0
   NaN   NaN     0

~~~~~> it is a stable system! <~~~~~

 Number of right hand side poles = 0

 Given polynomial coefficients roots :

sysRoots =

   0.6641 + 1.8230i
   0.6641 - 1.8230i
  -0.0000 + 2.0000i
  -0.0000 - 2.0000i
  -1.3283 + 0.0000i


ans =

     1     6     8
     0     5    20
  -Inf  -Inf     0
   NaN   NaN     0
   NaN   NaN     0
   NaN   NaN     0

\end{verbatim} \color{black}
    \begin{verbatim}
syms s
(s+1)*(s+2)*(s+3)*(s+4)
\end{verbatim}

        \color{lightgray} \begin{verbatim} 
ans =
 
(s + 1)*(s + 2)*(s + 3)*(s + 4)
 
\end{verbatim} \color{black}
    \begin{par}
syms a b c EPS;
\end{par} \vspace{1em}
\begin{verbatim}
syms a b c EPS;
ra=routh([1 a b c],EPS)
\end{verbatim}

        \color{lightgray} \begin{verbatim} 
ra =
 
[            1, b]
[            a, c]
[ -(c - a*b)/a, 0]
[            c, 0]
 
\end{verbatim} \color{black}
    

\subsection*{Question 12}

\begin{par}
K(s+2)/s(s-1)(s+3)
\end{par} \vspace{1em}
\begin{verbatim}
syms s K EPS;
ra2 = routh([1 2 K-3 2],EPS)
\end{verbatim}

        \color{lightgray} \begin{verbatim} 
ra2 =
 
[                 1, K - 3]
[                 2,     2]
[             K - 4,     0]
[ (2*K - 8)/(K - 4),     0]
 
\end{verbatim} \color{black}
    

\subsection*{Quesition 20}

\begin{verbatim}
ra3 = routh([K+1 3 2+K],EPS)
\end{verbatim}

        \color{lightgray} \begin{verbatim} 
ra3 =
 
[ K + 1, K + 2]
[     3,     0]
[ K + 2,     0]
 
\end{verbatim} \color{black}
    

\subsection*{Question 21}

\begin{verbatim}
ra4 = routh([1 5 4+K 6*K],EPS)
% solve(4 - K/5==0)
% solve((6*K*(K - 20))/(5*(K/5 - 4))==0)
\end{verbatim}

        \color{lightgray} \begin{verbatim} 
ra4 =
 
[                            1, K + 4]
[                            5,   6*K]
[                      4 - K/5,     0]
[ (6*K*(K - 20))/(5*(K/5 - 4)),     0]
 
\end{verbatim} \color{black}
    

\subsection*{Question 22}

\begin{verbatim}
syms a b c EPS;
ra5 = routh([1 K-b -a],EPS)
\end{verbatim}

        \color{lightgray} \begin{verbatim} 
ra5 =
 
[     1, -a]
[ K - b,  0]
[    -a,  0]
 
\end{verbatim} \color{black}
    

\subsection*{Example 7.2 from Nise}

\begin{verbatim}
syms s
G(s) = 120*(s+2)/((s+3)*(s+4))
CSlashR = simplify(G(s)/(1+G(s)))
polyVector = [1 127 252]
routh([1 127 252],EPS)
limit(G(s),s,0)
limit(s*G(s),s,0)

limit(s^2*G(s),s,0)
\end{verbatim}

        \color{lightgray} \begin{verbatim}Error using InputOutputModel/subsasgn (line 57)
Cannot assign a model of class "sym" into a model of class "tf".

Error in MidtermPrep (line 40)
G(s) = 120*(s+2)/((s+3)*(s+4))
\end{verbatim} \color{black}
    

\subsection*{Example 7.3 from Nise}

\begin{verbatim}
syms s
G(s) =100*(s+2)*(s+6)/(s*(s+3)*(s+4))
CSlashR = simplify(G(s)/(1+G(s)))
polyVector = [1 107 812 1200]
routh(polyVector, EPS)
%R(s) = 5*1/s
limit(5/(1+G(s)),s,0)
\end{verbatim}


\subsection*{Example 7.4 from Nise}

\begin{verbatim}
G(s) = 10*(s+20)*(s+30)/((s*(s+25)*(s+35)))
CSlashR = simplify(G(s)/(1+G(s)))
polyVector = [1 70 1375 600]
routh(polyVector, EPS)
\end{verbatim}
\begin{verbatim}
G(s) = 10*(s+20)*(s+30)/((s^2*(s+25)*(s+35)*(s+50)))
CSlashR = simplify(G(s)/(1+G(s)))
polyVector = [1 110 3875 43760 500 6000]
routh(polyVector, EPS)
\end{verbatim}
\begin{verbatim}
G1(s) = 500*(s+2)*(s+5)/((s+8)*(s+10)*(s+12))
G2(s) = 500*(s+5)*(s+5)*(s+6)/(s*(s+8)*(s+10)*(s+12))
G3(s) = 500*(s+2)*(s+4)*(s+5)*(s+6)*(s+7)/(s^2*(s+8)*(s+10)*(s+12))
Func1 = (simplify(G1(s)/(1+G1(s))))
Func2 = (simplify(G2(s)/(1+G2(s))))
Func3 = (simplify(G3(s)/(1+G3(s))))
\end{verbatim}
\begin{verbatim}
[n1,d1] = numden(Func1)
routh(sym2poly(d1), EPS)
[n2,d2] = numden(Func2)
routh(sym2poly(d2), EPS)
[n3,d3] = numden(Func3)
routh(sym2poly(d3), EPS)

Kp = limit(G1(s),s,0)
Kv = limit(s*G2(s),s,0)
Ka = limit(s^2*G3(s),s,0)
\end{verbatim}
\begin{par}
Try
\end{par} \vspace{1em}
\begin{verbatim}
numg=1000*[1 8];
deng=poly([-7 -9]);
G=tf(numg,deng);
Kp=dcgain(G)
estep=1/(1+Kp)
T=feedback(G,1);
poles=pole(T)
\end{verbatim}


\subsection*{Chapter 7 or 8 root Locus Plots}

\begin{par}
Root Locus Plot
\end{par} \vspace{1em}
\begin{verbatim}
h = tf([2 5 1],[1 2 3]);
rlocus(h)
snapnow;
\end{verbatim}
\begin{par}
Example from textbook
\end{par} \vspace{1em}
\begin{verbatim}
syms s
x = s*(s+1)*(s+2)*(s+4)
den = sym2poly(x)
num = (s+3)
num = sym2poly(num)
h = tf([num],[den])
rlocus(h)
\end{verbatim}
\begin{par}
$$ $$
\end{par} \vspace{1em}
\begin{verbatim}
x = (s+2)*(s+4)*(s+6)
den = sym2poly(x)
h = tf([1],[den])
rlocus(h)
\end{verbatim}
\begin{verbatim}
syms s K EPS
x =s^4+ 7*s^3+14*s^2+(8+K)*s+3*K
polyVec = [1 7 14 8+K 3*K]
xD = routh(polyVec, EPS)
xD = simplify(xD)
\end{verbatim}
\begin{verbatim}
polyVec = [1 6 8]
routh(polyVec,EPS)
% stable function
rlocus(tf([1 4 20],[1 6 8]))
\end{verbatim}
\begin{par}
Question 8--4
\end{par} \vspace{1em}
\begin{verbatim}
num= tf([1 2/3])
den = tf([1 6 0 0])

sys = tf([1 2/3], [1 6 0 0])
rlocus(sys)
\end{verbatim}



\end{document}
    
